\documentclass[letterpaper,11pt]{report}
% Change margins to 1 inch on all sides
\addtolength{\oddsidemargin}{-.875in}
\addtolength{\evensidemargin}{-.875in}
\addtolength{\textwidth}{1.75in}
\addtolength{\topmargin}{-.875in}
\addtolength{\textheight}{1.75in}
\usepackage{float}
\usepackage{graphicx}
\usepackage{footnote}
\usepackage{longtable}
\usepackage{multirow}
\usepackage{tablefootnote}
\usepackage{tabularx}
\usepackage{url}
\DeclareGraphicsExtensions{.pdf,.png,.jpg}

%%%%%%%%%%Start of report
\begin{document} 
\begin{savenotes}
\pagestyle{plain}
\title{CS896 Introduction to Web Science\\Fall 2013\\Report for Assignment 10}
\author{Corren G. McCoy}
 
\date{December 12, 2013}
\maketitle

\renewcommand*\thesection{\arabic{section}}
\setcounter{section}{0}

\setcounter{tocdepth}{4}
\tableofcontents
 \listoftables
\newpage


%%%%%%%%%%Chapter Exercises
\section{Question 1}
\subsection{Problem}Choose a blog or a newsfeed (or something similar with an Atom or RSS feed).  It should be on a topic or topics of which you are  qualified to provide classification training data.  Find something with at least 100 entries. Create between four and eight different categories for the entries in the feed. Download and process the pages of the feed as per the week 12 class slides.

\subsection{Response}For our blog, we chose Jingle Bell Junction (\url{http://jinglebelljunction.com/}), which is self-described as ``the merriest Christmas site on the web!'' The content of the blogs and d downloads consist of holiday recipes, news \& articles, crafts, homemade gift ideas, and various other Christmas-related topics. After reviewing the blogs, we chose the following categories:
\begin{itemize}
\item recipe;
\item craft;
\item activity;
\item gift idea; and
\item news article.
\end{itemize}
To ensure consistency as we are attempting to classify the entries, the RSS feed was downloaded to an XML file (i.e., jinglebell.xml). The XML file is included in the supporting documents for this assignment.

%%%%%%%%%%Chapter Exercises
\section{Question 2}
\subsection{Problem}Manually classify the first 50 entries, and then classify (using the fisher classifier) the remaining 50 entries. Report the cprob() values for the 50 titles as well.  From the title or entry itself, specify the 1-, 2-, or 3-gram that you used for the string to classify.  Do not repeat strings; you will have 50 unique strings. Create a table with the title, the string used for classification, cprob(), predicted category, and actual category.
\subsection{Response}We used the docclass.py and feedfilter.py files found in Segaran \cite{segaran2007programming} as the basis for our document filtering. The modified source code, shown in Appendices \ref{chap:pythona} and \ref{chap:pythonb}, performs the following tasks:

\begin{itemize}
\item Use \emph{doclass.py} to extract features from the title and summary from the first 50 entries in the RSS feed. 
\item Remove HTML tags before dividing the remaining text of each entry into individual words.
\item Manually train the classifier using the Fisher method. Save the features and related categories to a set of database tables so the training will persist between sessions.
\item Use \emph{feedfilter.py} to parse the title and summary of the RSS feed for the remaining 50 entries.
\item Use the Fisher method to predict a category based on the entry.
\item Prompt the user to enter the actual category along with an n-gram to determine the Fisher probability (i.e. cprob()).
\item Save the entry title, feature, predicted category, actual category, and cprob() to database table.
\end{itemize}

Table \ref{tab:training} shows the number of entries that were allocated to each of defined categories while training the classifier.

\begin{table}[htbp]
	\centering
    \begin{tabular}{|p{4cm}|l|}
    \hline
    Category & Entries \\ \hline
    craft    & 15    \\ \hline
    recipe   & 4     \\ \hline
    activity & 11    \\ \hline
    news     & 24    \\ \hline
    gift     & 6     \\ \hline
    \end{tabular}
    \caption {Entries per Training Category}
			\label{tab:training}
\end{table}

The results of classifying the Jinglebell Junction RSS feed are shown in Table \ref{tab:results}. Based on the overwhelming number of probabilites rated at 0, we can determine the classifier did not perform well on this particular data. As stated in Segaran \cite{segaran2007programming}, this might be attributed to the unequal number of documents allocated to each category during training. We can see from Table \ref{tab:training} that blogs in Jinglebell Junction skew more heavily towards news articles and craft ideas.

\begin{table}[htbp]
	\centering
    \begin{tabular}{|p{8cm}|l|l|l|l|}
    \hline
    \textbf{Title} & \textbf{Feature} & \textbf{cprob()} & \textbf{Predicted} & \textbf{Actual} \\ \hline
Poinsettia Fact and Fiction&belief&1.0&news&news\\ \hline
Reindeer Hokey Pokey&hokey pokey&0.0&news&activity\\ \hline
Christmas Punch&punch&0.0&craft&recipe\\ \hline
Holiday Pumpkin Cheesecake&cheesecake&0.0&craft&recipe\\ \hline
Chocolate Chip Toffee M\&M Cookies&cookies&0.0&news&recipe\\ \hline
Pumpkin Walnut Fudge&fudge&0.0&recipe&recipe\\ \hline
Peanut Brittle&peanut&0.0&gift&recipe\\ \hline
Peanut Butter Cups&butter&0.0&news&recipe\\ \hline
Double Chocolate Caramels&chocolate&0.5&recipe&recipe\\ \hline
Easiest Fudge in the World!&ingredient&0.0&recipe&recipe\\ \hline
Jinglebelle’s Pumpkin Pancakes&pancake&0.0&recipe&recipe\\ \hline
Spiced Pumpkin Fudge&spiced&0.0&recipe&recipe\\ \hline
Jinglebelle’s Chocolate Ice Cream&ice cream&0.0&news&recipe\\ \hline
Hot Russian Tea&tea&0.0&news&recipe\\ \hline
Scented Gel Air Fresheners&scented&0.0&activity&craft\\ \hline
Paint Stirrer Snowman&paint&0.0&activity&craft\\ \hline
Craft Stick Angel&stick&0.0&craft&craft\\ \hline
The Origins of Santa&santa&0.234&news&news\\ \hline
Snowman Soup Hot Chocolate&soup&0.0&recipe&recipe\\ \hline
Decorated Clay Ornaments&clay&0.0&news&craft\\ \hline
Christmas Sponge Art&sponge&0.0&recipe&craft\\ \hline
Reindeer Candycane Ornament&ornament&0.0&recipe&activity\\ \hline
Chocolate Melting Spoons&spoons&0.0&recipe&craft\\ \hline
How to find your screen resoultion&screen&0.0&news&news\\ \hline
Installing Christmas wallpapers on your iPhone&wallpapers&0.0&news&news\\ \hline
Santa Hat Gift Tags&tags&0.0&gift&craft\\ \hline
How to cook a perfect Thanksgiving turkey&turkey&1.0&news&activity\\ \hline
Craft Stick Angel&craft&1.0&craft&craft\\ \hline
Jingle Bell Napkin Rings&napkin&0.0&craft&craft\\ \hline
2011 Christmas Expo Lights Up Gatlinburg, TN This Summer&expo&0.0&news&activity\\ \hline
``My Favorite Gift''by Virginia Blanck Moore&favorite gift&0.0&news&news\\ \hline
Starting \& Adding to Your Christmas Music Library – A Primer&music&0.0&news&activity\\ \hline
Simple Techniques for Keeping Your Child Believing in Santa Claus&simple techniques&0.0&news&news\\ \hline
Embossed Velvet Stockings&velvet&0.0&news&craft\\ \hline
Snowman Clip Art&clip art&0.0&news&craft\\ \hline
Santa Claus Clip Art&graphics&0.0&news&craft\\ \hline
Grinch Coloring&grinch&0.0&craft&activity\\ \hline
Christmas House Clip Art&house&0.125&news&craft\\ \hline
A Jinglebell Junction Exclusive!! Two Trees&two trees&0.0&news&activity\\ \hline
Jar Lid Magnets&magnets&0.0&activity&craft\\ \hline
		\end{tabular}
	\caption{Classification Results}
	\label{tab:results}
\end{table}


%%%%%%%%%%Chapter Exercises
\section{Question 3}
\subsection{Problem}Assess the performance of your classifier in each of your categories by computing precision and recall.  Note that the definitions are slightly different in the context of classification; see: \small\url{http://en.wikipedia.org/wiki/Precision_and_recall#Definition_.28classification_context.29}\par
\subsection{Response}Since our data is classified using multiple categories, we will use a confusion matrix (\small\url{http://en.wikipedia.org/wiki/Confusion_matrix}) to analyze the results and calculate precision and recall. The matrix as shown in Table \ref{tab:confusion} will illustrate how well the classifier was able to make the correct predictions. Based on the entries in the confusion matrix, we can now compute precision and recall as shown in Table \ref{tab:performance}. We can see that entries in the \emph{recipe} category have the highest level of precision, while entries in the \emph{news} category have the highest level of recall.

\begin{table}[htbp]
\centering
    \begin{tabular}{|p{4cm}|l|l|l|l|l|}
    \hline
						 & \multicolumn{5}{|c|}{Predicted Class} \\ \hline
    Actual   & recipe & craft & gift & news & activity \\ \hline
    recipe   & 6      & 2     & 1    & 4    & 0       \\ \hline
    craft    & 2      & 3     & 1    & 5    & 3       \\ \hline
    gift     & 0      & 0     & 0    & 0    & 0       \\ \hline
    news     & 0      & 0     & 0    & 6    & 0       \\ \hline
    activity & 1      & 1     & 0    & 5    & 0       \\ \hline
    \end{tabular}
    \caption {Confusion Matrix}
			\label{tab:confusion}
\end{table}

\begin{table}[htbp]
\centering
    \begin{tabular}{|p{4cm}|l|l|}
    \hline
    ~        & Precision & Recall \\ \hline
    recipe   & 6/9   (0.67)  & 6/13 (0.46)  \\ \hline
    craft    & 3/6   (0.50)  & 3/14 (0.21)  \\ \hline
    gift     & 0/2   (0.00)  & 0/0  (0.00)  \\ \hline
    news     & 6/20 (0.30)  & 6/6  (1.00)  \\ \hline
    activity & 0/3   (0.00)  & 0/7  (0.00)  \\ \hline
    \end{tabular}
    \caption {Performance Measures}
		\label{tab:performance}
\end{table}
%%%%%%%%%%Chapter Exercises
\section{Question 4 Extra Credit}
\subsection{Problem}Redo the questions above, but with the extensions on slide 26 and pp. 136--138.
\subsection{Response}Not attempted.

\end{savenotes}

% produce the bibliography for the citations in your paper.
\bibliographystyle{abbrv}
\bibliography{cmccoy}

\appendix
\addcontentsline{toc}{chapter}{Appendices}

%%Appendix A
\chapter{Python Source Code - docclass.py} \label{chap:pythona}
\input{docclass.py}
\chapter{Python Source Code - feedfilter.py} \label{chap:pythonb}
\input{feedfilter.py}


\end{document} 
%%%%%%%%%%Ed of report
